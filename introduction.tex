\section{Introduction}
This section presents related work to a slacklining assistance system with an interactive technology approach. It provides exercises and feedback for beginners on a slackline with the Microsoft Kinect v2 as a tracking device. Hence it is necessary to have a good overview about existing systems, approaches and studies that can help to build an appropriate concept and system. First related concepts regarding slacklining show how to build learning techniques for beginners, the efficacy of it in balance training, and areas of application. Next current tracking technologies have to be compared for tracking the human body on the slackline and why the Microsoft Kinect v2 seems like the best tracking device. Lastly the system has to be aware of cognitive load and motivating aspects, which can be challenging with repetitive exercises. Several applications show where problems occur with different feedback and interaction methods. Also design opportunities for guiding the user through the learning process are demonstrated by various approaches.